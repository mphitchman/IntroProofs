\documentclass{article}
\usepackage{amssymb}
\pagestyle{empty}

\setlength{\topmargin}{-.5in}
\setlength{\oddsidemargin}{-.25in}
\setlength{\textwidth}{6in}
\setlength{\textheight}{9in}

\everymath{\displaystyle}

\begin{document}


{\bf Math 220}\hfill{Name:}

{\bf Homework 4: Proof by Contradiction, Proof by Contrapositive}
\vspace{.3in}

Write all proofs clearly and neatly on your own paper; ie, your proofs should contain complete sentences!!! Remember: your proofs should strive to be correct, clear, and complete.


\begin{enumerate}
\item  Use proof by cases for the following problems.

\begin{enumerate}
\item Prove that the product of any two consecutive integers is even.

\smallskip

\item Prove that the square of any integer has the form $3k$ or $3k+1$ for some integer $k$.

\smallskip

\item Prove that for any integer $n$, $n(n^2-1)(n+2)$ is divisible by 4.

\smallskip

\end{enumerate}

\item Prove the following by contradiction.

\begin{enumerate}
\item There is no greatest even integer.

\smallskip

\item There is no least positive rational number.

\smallskip

\end{enumerate}
\item Prove the following by contrapositive.

\begin{enumerate}
\item If the product of two positive real numbers is greater than 100, then at least one of the numbers is greater than 10.

\smallskip

\item If the sum of two real numbers is less than 50 then at least one of the numbers is less than 25.

\smallskip

\end{enumerate}

\item Prove the following statements by either contradiction or contrapositive (be sure to note which method you used).
\begin{enumerate}
\item For all integers $n$, if $n^2$ is odd then $n$ is odd.

\smallskip

\item For all integers $n$ and all prime numbers $p$, if $n^2$ is divisible by $p$, then $n$ is divisible by $p$.
\smallskip
\item For all integers $a$, $b$, and $c$, if $a$ does not divide $(bc)$, then $a$ does not divide $b$.
\smallskip

\item For all integers $a$, $b$, and $c$, if $a$ divides $b$ and $a$ does not divide $c$, then $a$ does not divide $(b+c)$.
\smallskip
\end{enumerate}

\item Determine whether the following statements are true or false. Prove the true statements by contradiction and provide counterexamples for the false ones.
\begin{enumerate}
\item The sum of any two irrational numbers is irrational.
\smallskip
\item If $a$ and $b$ are rational numbers, $b\neq 0$, and $r$ is an irrational number, then $a+br$ is irrational.
\smallskip
\item If $r$ is any rational number, and $s$ is any irrational number, then ${r\over s}$ is irrational.
\smallskip
\end{enumerate}

\item If $a$, $b$, and $c$ are integers and $a^2+b^2=c^2$, must at least one of $a$ and $b$ be even? Justify your answer.
\bigskip

\item From Worksheet 1 we know that, given any statement of truth-functional logic, we can always find a logically equivalent statement that uses only ``$\neg$" and ``\&". For example, $[(p\leftrightarrow q) \vee r]$ is logically equivalent to 
$\neg\{\neg r\ \&\ [\neg (\neg p\ \&\ \neg q)\ \&\ \neg (p\ \&\ q)]\}$. So we could, if we wanted to, do without the connectives ``$\vee$", ``$\underline \vee$", ``$\rightarrow$", and ``$\leftrightarrow$". Suppose that, instead, we chose to use {\it only} the connectives ``\&", ``$\vee$", ``$\rightarrow$", and ``$\leftrightarrow$". Would these connective suffice? Why or why not? Remember that whatever your response, you should be able to offer a {\it proof} of it!
\medskip

\item Consider the following statement and ``proof." Is the proof correct? If so, what proof strategies does it use? If not, can it be fixed? Is the theorem correct?

 ``{\bf Theorem.}" There are irrational numbers $a$ and $b$ such that $a^b$ is rational.
\smallskip
\begin{enumerate}
\item[] \underbar{Proof.} Either $\sqrt{2}^{\sqrt{2}}$ is rational or it is irrational.
\smallskip
\begin{enumerate}
\item[]  Case 1: $\sqrt{2}^{\sqrt{2}}$ is rational. Let $a=b=\sqrt{2}$. Then $a$ and $b$ are irrational, and $a^b=\sqrt{2}^{\sqrt{2}}$, which we assumed was rational.
\smallskip
\item[]  Case 2: $\sqrt{2}^{\sqrt{2}}$ is irrational. Let $a=\sqrt{2}^{\sqrt{2}}$ and $b=\sqrt{2}$. Then $a$ is irrational by assumption, and we know that $b$ is also irrational. Furthermore, $a^b=\bigl(\sqrt{2}^{\sqrt{2}}\bigr)^{\sqrt{2}}=\sqrt{2}^{(\sqrt{2}\sqrt{2})}=2$. Which is rational.
\end{enumerate}
\end{enumerate}


\end{enumerate}
\end{document}